\documentclass{article}%
\usepackage{amsmath}%
\usepackage{amsfonts}%
\usepackage{amssymb}%
\usepackage{graphicx}
\usepackage[ruled,vlined]{algorithm2e}
%-------------------------------------------
\newtheorem{theorem}{Theorem}
\newtheorem{acknowledgement}[theorem]{Acknowledgement}
\newtheorem{axiom}[theorem]{Axiom}
\newtheorem{case}[theorem]{Case}
\newtheorem{claim}[theorem]{Claim}
\newtheorem{conclusion}[theorem]{Conclusion}
\newtheorem{condition}[theorem]{Condition}
\newtheorem{conjecture}[theorem]{Conjecture}
\newtheorem{corollary}[theorem]{Corollary}
\newtheorem{criterion}[theorem]{Criterion}
\newtheorem{definition}[theorem]{Definition}
\newtheorem{example}[theorem]{Example}
\newtheorem{exercise}[theorem]{Exercise}
\newtheorem{lemma}[theorem]{Lemma}
\newtheorem{notation}[theorem]{Notation}
\newtheorem{problem}[theorem]{Problem}
\newtheorem{proposition}[theorem]{Proposition}
\newtheorem{remark}[theorem]{Remark}
\newtheorem{solution}[theorem]{Solution}
\newtheorem{summary}[theorem]{Summary}
\newenvironment{proof}[1][Proof]{\textbf{#1.} }{\ \rule{0.5em}{0.5em}}
\setlength{\textwidth}{7.0in}
\setlength{\oddsidemargin}{-0.35in}
\setlength{\topmargin}{-0.5in}
\setlength{\textheight}{9.0in}
\setlength{\parindent}{0.3in}
\begin{document}

\begin{flushright}
\textbf{Brandon Toner \\
\today}
\end{flushright}

\begin{center}
\textbf{CS 405: Algorithm Analysis II \\
Homework 7: Greedy Algorithms} \\
\end{center}

\begin{enumerate}
\item The $N$th floating point number will either be in one of the intervals found from processing the first $N-1$ numbers, or a new interval.  This algorithm assumes that if $G_N \in [v.max - 1, v.min + 1]$ then $G_N$ is in one of the previous intervals, and there is no need to check any more intervals.\\
\begin{algorithm}
    \DontPrintSemicolon
    \KwData{$G, N$}
    \Begin{
        \If{$N=1$}{
            \Return{$(min=G_0, max=G_0)$}
        } \Else {
            $data=UnitIntervals(G, N-1)$\\
            \For{$v \in data$}{
                \If{$G_N \leq v.min + 1$ \bf{and} $G_N \geq g.max - 1$} {
                    \If{$G_N > v.max$} {
                        $v.max=G_N$
                    }
                    \If{$G_N < v.min$} {
                        $v.min=G_N$
                    }
                    \Return $data$
                }
            }
           \Return $data + (min=G_N, max=G_N)$
        }
    }
\caption{UnitInterval\label{IR}}
\end{algorithm}

\item This algorithm assumes that since multiplication is commutitiave, we can rearrange the product in a way that the fist element in the product will be the largest. Therefore the greedy choice for the first value will be to choose $i,j$ such that ${a_i}^{b_j}$ is a maximum. This value will be $a_m^{b_m}$ where $a_m = max(A)$ and $b_m = max(B)$, since decreaing the exponent or the base would decrease the value.  As for the optimal substructure, if we assume we have a perumtation, $S$ that is the optimal solution, and a subproblem $p \subset S$ p must also be an optimal solution.  If there were another permutation of the exponents that yeilded a larger product, we would be able to repalce the elements of $p$ in $S$ to make a larger solution, which is a contradiction that $S$ is the optimal solution.\\
\begin{algorithm}
    \DontPrintSemicolon
    \KwData{$A, B$}
    \Begin{
        \If{$len(A)=1$}{
            \Return{$pow(a_0, b_0)$}
        } \Else {
            $a_m = max(A)$\\
            $b_m = max(B)$\\
            \Return {$pow(a_m, b_m) * MaxProduct(A - a_m, B - b_m)$}
        }
    }
\caption{MaxProduct\label{IR}}
\end{algorithm}


\end{enumerate}


\end{document}
